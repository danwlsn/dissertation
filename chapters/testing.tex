\subsection{Introduction}
This chapter will cover the testing of application. It will go into detail about the choices made and give reasons as to why certain tools or methodologies were used over others.\\

\subsection{Methodologies}
For the main part of testing, test driven development(TDD) was used throughout the building of the application. TDD was used because of the constant deployments to Heroku. All of the features implemented had to be working perfectly before deployment. Sometimes this was not the case though. All of the tests are written so sometimes you miss a test case and this can lead to deploying broken code. Usually, most of the bugs were picked up either by the testing or quite quickly after deployment if any bugs got through. TDD fitted in well with the Agile development approach that was taken when building the application. TDD and Agile development were used to make testing easy. TDD tests the code as you are working so you are not left doing all the test last, or after every feature. This allowed to get working prototypes out and deployed very quickly.\\

\subsection{Tools}
\subsection{User testing}
User testing was submitted by the users of the application.
A bug that was picked up by a user was that the fitness log was pulling out everyones fitness logs, not just the users. It was simply a typo in the controller that was pulling out every fitness log in the database, as apposed the specific users fitness log.
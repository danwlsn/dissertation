\section{Introduction}
This chapter will cover the testing and evaluation of application. It will go into detail about the choices made and give reasons as to why certain tools or methodologies were used over others. It will also discuss the first release and reviewed release of the application, and talk about the user's feedback on each release.\\

\section{Methodologies}
For the main part of testing, test driven development(TDD) was used throughout the building of the application. TDD was used because of the constant deployments to Heroku. All of the features implemented had to be working perfectly before deployment. Sometimes this was not the case though. All of the tests are written so sometimes you miss a test case and this can lead to deploying broken code. Usually, most of the bugs were picked up either by the testing or quite quickly after deployment if any bugs got through. TDD fitted in well with the Agile development approach that was taken when building the application. TDD and Agile development were used to make testing easy. TDD tests the code as you are working so you are not left doing all the test last, or after every feature. This allowed to get working prototypes out and deployed very quickly.\\

\section{Tools}
\subsection{Rspec}
\subsection{Chrome Development Tools}
Chrome Developer Tools \citep{cdt:2014} was a key part to the testing of the layout of this web application. Using the developer tools you are able to target specific parts of the application and see where they are getting their CSS styles from. This was especially useful when reusing class selectors on the cards. The cards would share similar styles but also have some individual styles that would conflict if kept under one class. It was also very good for mobile testing. It also allows for on the fly changing of CSS selectors and values. This proved useful for find ways to fix certain layout issues as it does not save the changes. The Chrome Developer Tools has an emulation mode, which allows the user to see what the website would look on a selection of devices. This proved more usefully than just resizing the browser window as it took viewport into account. \\

\section{First release}
March 11th 2014 marked the first release of recur. It was posted to the fitness subreddit of the popular image board site Reddit\citep{reddit:2005}. The idea was to do a small release to pick up on any bugs that got through testing. It became very helpful because as the developer of the site knew how everything worked because they had built it, other people said they did not know where to click. This subject had been raised by colleague and fixed with an image of a chain link to denote a link. After this feedback had been submitted, changes were made to the title links. They were made more obvious when rolled over with a background colour change and also, if it is the users first visit to the site then they will be shown a message telling them that the titles of the cards are links to more information. Other feedback included users receiving error flashes when really the database was successfully updated. This was because the error flash was getting called if the log entry was saved or not. This caused uses to submit multiple entries to the database for the same activity. This then caused another problem of not being able to removed a logged activity after it was put into the database. An obvious feature was overlooked during development but was soon implemented in to make sure users could delete unwanted logged activities. The way the log was displayed was also rewritten to make it easier to see which data went with which activity. Users also wanted to be able to see which goals they had completed on the dashboard. This was limited down to goals that had been completed in the last 2 weeks.

\section{Design changes}
Along the progression of development, although nothing major has changed, the design has changed in a number of ways for a number of different reason. Probably the biggest thing that has changed during development was the colour of the website. Originally, it was a shade of blue with some red in it to create this dark lavender colour. It has now been changed to a shade of blue that favours green end. The change was purely aesthetic but I feel it was the better choice. The small chart on the front page was dropped quite early on during development. It was very had to implement such a stripped down chart that did not display very much information upon the quick glances it would receive. It was swapped out with a much large, more descriptive chart on the dedicated weights page. Now the chart is one more click away but it allows the chart more space to display more information and , therefore, becoming more useful to the user.


\section{Conclusion}
\section{Learnings}
The outcome of building this design has been a great help for me personally. It gave me the opportunity to learn a lot about what goes into an application that has the potential to gain lots of users. The design process taught me lots about designing for user accessibility. Before, when designing my application, it was all about appearance. But now I understand that there is more to web design than just aesthetics. It is important that the user is able to use the application easily without much hand holding. The design should ease users towards where they need to go, not just expect them to know exactly how to use it straight away. Also, some users may be visually impaired. This has the potential to cause a lot of problems when they're using the website. If you have not catered for them, it could be impossible for them to interact with the website. This is a major problem as you are alienating a large percentage of the population. On a similar note, being able to support a multitude of browsers and devices is important. As mentioned in the report, with mobile traffic increasing it is very important that the application has the same functionality on mobile as it does in the browser. And across all browsers, not just the main ones. The last thing I learnt was that mock ups are important. I never used to think that they were and I would just jump straight into the browser, but with this project I found that they are a great help to have something to refer to instead of having to think up where you are going to put element X.\\

Development taught me that having an understanding of the tools you are going to be using is very important. I went in with very little knowledge of Ruby on Rails and my project was set back many weeks because of it. I will, from now on, make sure that I have a good understanding of everything that will be involved with my project before I undergo any serious development. The pause in the middle of development to go away and learn the language was a massive set back. If I could go back and do this again I would spend the first few weeks gaining knowledge of the tools and technologies I would be using. Also, I had to learn about security and keeping my users safe from attackers. I was lucky enough to have a lot of the functionality from other tools (bcrypt, PostgreSQL hosted on Heroku) that I did not need to worry about it so much, but it was still something I had to take into consideration.\\

During testing I found that it is very important to test from the very beginning. I was not using test driven development at the beginning and it was taking a very long time to make progression because I was finding bugs all the time in previously written code. Once I implemented test driven development, progression sped up and I was able to rapidly progress through and get a releasable product. I also learnt that you can only test so much. Even thought I had been testing thoughout development, upon first release almost every user had found a bug. Whilst this was not ideal, it was certainly helpful that they reported it. At first, I was using the built in Rails testing framework which I found to be restricted and not as simple as what I had seen form Rspec. As soon as I switched over to Rspec I found that I was writing test much quicker and it was easier to debug when an error was found. If I was to start again, I would definitely use Rspec from the beginning.

\section{Future plans}
I plan to maintain Recur for the foreseeable future. I feel it has potential to become something useful for a broad portion and a variety of society. The application could use a new design. Whilst I am satisfied with the design and I think that it for fills its purpose, it definitely needs reworking in some parts. Also, some of the code could use refactoring. While this is not a major issue, if I was to ever sell this on to someone else, having a clean, commented, and well documented code base is very important. \\

Another project that could happen with this application is the development of an iOS or Android application. One of the great aspects Ruby on Rails is that it is very easy to extend the functionality of the web application to an API. This will then allow me to develop a mobile application that will make calls to Recur's API so the users will be able to retrieve and update their information from a native mobile application. Seeing as one of the main goals is to make sure that the application works in mobile browsers and the main use of this application is to update your fitness log on the go, it would make sense for a native mobile application to exist.
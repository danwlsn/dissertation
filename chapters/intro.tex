\section{Fitness applications}
Fitness applications are important because they provide people with a way to monitor how they are getting on with their health. Keeping fit and getting healthy is an everlasting venture, and improvements are only noticeable over a long period of time. Although, whilst the participant may not see a physical change, their health and fitness is improving with every visit to the gym. A fitness monitoring application allows the user to track what activities they completed at the gym, and how well they performed. For example, if you went on the treadmill, you could track how long it took you to run a mile. Next time you go on the treadmill, you will log your mile time again and see if you have improved. It also gives that edge of motivation. By showing you how well you are doing it gives you the motivation to keep at it. If you are not doing so great, it will allow you to see where you need to improve.\\

\section{Benefits of being fit and healthy}
Scientific studies and statistics provide evidence that obesity is a growing problem in modern society. It has been on the rise for a long time. Obesity in adult men was up to 35.5\% and women up to 35.8\% in 2009-2010 \citep{doi:10.1001}. This is obviously not good at all. This is a problem that must be tackled. There are many benefits of leading a fit and healthy lifestyle. A number of which are \citep{nhs:2013}:

\begin{enumerate}
\item {Up to a 35\% lower risk of coronary heart disease and stroke.}
\item {Up to a 50\% lower risk of type 2 diabetes.}
\item {Up to a 30\% lower risk of depression.}
\item {Up to a 30\% lower risk of dementia.}
\end{enumerate}

\section{Applications}
Mobile and web applications are on the rise. People are constantly on their phones interacting with their applications, be it a native mobile application or a web based application. Applications are useful because they usually perform one or two tasks really well. Be that recording notes or tracking your blood sugar levels. They allow the user to track information and keep it all organised. With this information, the application will be able to display back to the user in an orderly fashion; either in the form of a table listing all their previous expenses, or a graph showing how much they have spent in certain categories.

\section{Recur}
Recur is the fitness application that is being documented in this dissertation. It focuses on three main aspects of fitness. These are:

\begin{enumerate}
\item {Current and goal weight}
\item {Fitness specific goals}
\item {Fitness log}
\end{enumerate}

\noindent
By keeping the application down to these three features, it will allow the users to maintain focus and not become distracted by other, unnecessary aspects. It will be displayed in a modern, minimalistic design and built using up to date technologies (this will be explored further in the development Chapter \ref{sec:dev} on Page \pageref{sec:dev}).\\

\section{Learnings}
Throughout the production of this application, there will a number of opporunities where knowledge can be gained. The design stage will have to take into account user accessibility. This covers catering the design for the visually impaired and making the website as user friendly as possible. The development stage will cover the learnings of using a new MVC framework, building and authentication system, and securing it from attacks. Testing will teach the why testing is important and how to be efficient when writing tests.\\

\section{Dissertation structure}
This dissertation will cover the entire process behind the building of the progress monitoring web application. The application is for monitoring the user's fitness progress. It shall allow a user to track their current and goal weight, any goals they may want to achieve, and a diary of fitness entries. The application will focus on these three main points so as not to not distract the user away from their progress. Other features like social media sharing and calorie tracking will not be implemented. More information on these choices can be found in the design Chapter \ref{sec:design} on Page \pageref{sec:design} of this paper. The literature review Chapter \ref{sec:litrev} on Page \pageref{sec:litrev} will focus on the current state of fitness progress monitor applications that are already available for use. It will discuss both the good and bad aspects of each application and highlight the areas which will be improved upon or performed differently by the new application. The design Chapter \ref{sec:design} on Page \pageref{sec:design} will cover the reasoning behind most of the design choices that were made during the planning of this application. For example, which font was chosen and why. The design chapter will also cover some development choices, such as the data structure design. Following this, the resources Chapter \ref{sec:resources} on page \pageref{sec:resources} will talk about what tools, languages, and frameworks were used to develop the applicatoin. After that, the development Chapter \ref{sec:dev} on Page \pageref{sec:dev} will talk through the development process. It will go into detail regarding the problems and challenges that were faced during development and how they were overcome, and also the deployment of the web application to the internet. Following the development Chapter will be the testing Chapter \ref{sec:test} on Page \pageref{sec:test}. This will go over the testing methodologies and tools used in the building of the application and why they were used. It will also cover how bugs and errors picked up by the testing were overcome. Also covered in the testing Chapter will be the evaluation. The evaluation section will cover what the outcome of the product was in comparison to the original idea. It will also cover feedback from users about their experience and how their feedback will be acted upon. The conclusion (Chapter \ref{sec:conc} on Page \pageref{sec:conc}) will conclude the paper by demonstrating what knowledge has been gained from this build, what would have been done differently, and also possible future developments from the product.\\


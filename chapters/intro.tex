\section{Fitness applications}
Fitness applications are important because it gives people a way to monitor how they are getting on with their health. Keeping fit and getting healthy is an everlasting venture, and you will only notice improvements over a long period. Although, while you may not see a physical change, you are improving with every trip to the gym. What fitness monitor applications do is allow you to track what did at the gym and how well you did. For example, if you went on the treadmill, you could track how long it took you to run a mile. Next time you go on the treadmill, you will log your mile time again and see if you have improved. It also gives that edge of motivation. By showing you how well you are doing it gives you the motivation to keep at it. If you are not doing so great, it will allow you to see where you need to improve.\\

\section{Benefits of being fit and healthy}

\section{Applications}
Mobile and web applications are on the rise. People are constantly on their phones interacting with their applications, be it a native mobile application or a web based application. Applications are usefully because they usually do one or two tasks really well. Be that; recording notes or tracking your blood sugar levels. They allow the user to track information and keep it all organised. With this information, the application will be able to display back to the user in an orderly fashion; whether that is in the forma of table listing all their previous expenses, or a graph showing how much they have spent in certain categories.

\section{Recur}
Recur is the fitness application that is being documented in this dissertation. It focuses on three main aspects of fitness. These are:

\begin{enumerate}
\item {Current and goal weight}
\item {Fitness specific goals}
\item {Fitness log}
\end{enumerate}

\noindent
By keeping the application down to these three features, it will allow the users to keep focused and not get distracted by other, unnecessary aspects. It will be displayed in a modern, minimalistic design and built using up to date technologies (more on this in the development chapter \ref{sec:dev} on page \pageref{sec:dev}).\\

\section{Learnings}
A number of things will be learnt during the building of this application. The design stage will have to take into account user accessibility. This covers catering you design for the visually impaired and making the website as user friendly as possible. The development chapter cover the learnings of using a new MVC framework, building and authentication system, and securing it from attacks. In the testing chapter, it teaches how to write tests for test driven development.\\

\section{Thesis structure}
This dissertation will cover the whole process behind the building of the progress monitoring web application. The application is for monitoring the users fitness progress. It shall allow a user to track their current and goal weight, any goals they may want to achieve, and a diary of fitness entries. The application will focus on these three main points only to not distract the user away from their progress. Other features like social media sharing and calorie tracking will not be implemented. More information on these choices can be found in the design chapter \ref{sec:design} on page \pageref{sec:design} of the paper. The literature review chapter \ref{sec:litreview} on page \pageref{sec:litreview} will focus on current state of fitness progress monitor applications that are already available for use. It will discuss both the good and bad points about each application and what this application will do differently. The design chapter \ref{sec:design} on page \pageref{sec:design} will also cover the reasoning behind most of the design choices that were made during the planning of this application. For example, which font was chosen and why. The design chapter will also cover some development choices, such as the data structure design. After that, the development chapter \ref{sec:dev} on page \pageref{sec:dev} will talk through the development process. It will go into detail about which tools were used and why, the problems and challenges that were faced during development and how they were over come, and also deployment of the web application to the internet. Following the development chapter will be the testing chapter \ref{sec:test} on page \pageref{sec:test}. This will go over the testing methodologies and tools used in the building of the application and why they were used. It will also cover how bugs and errors picked up by the testing were overcome. The evaluation chapter \ref{sec:eval} on page \pageref{sec:eval} will cover what the outcome of the product was like compared with the original idea. It will also cover feedback from users about their and how that feedback will be acted upon. The conclusion chapter \ref{sec:conc} on page \pageref{sec:conc} will wrap up the paper with what knowledge has been gained from this build, what would have been done differently, and also possible future outcomes from the product.\\


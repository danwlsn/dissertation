This dissertation will cover the whole process behind the building of the progress monitoring web application. The application is for monitoring the users fitness progress. It shall allow a user to track their current and goal weight, any goals they may want to achieve, and a diary of fitness entries. The application will focus on these three main points only to not distract the user away from their progress. Other features like social media sharing and calorie tracking will not be implemented. More information on these choices can be found in the design chapter \ref{sec:design} on page \pageref{sec:design} of the paper. The literature review chapter \ref{sec:litreview} on page \pageref{sec:litreview} will focus on current state of fitness progress monitor applications that are already available for use. It will discuss both the good and bad points about each application and what this application will do differently. The design chapter \ref{sec:design} on page \pageref{sec:design} will also cover the reasoning behind most of the design choices that were made during the planning of this application. For example, which font was chosen and why. The design chapter will also cover some development choices, such as the data structure design. After that, the development chapter \ref{sec:dev} on page \pageref{sec:dev} will talk through the development process. It will go into detail about which tools were used and why, the problems and challenges that were faced during development and how they were over come, and also deployment of the web application to the internet. Following the development chapter will be the testing chapter \ref{sec:test} on page \pageref{sec:test}. This will go over the testing methodologies and tools used in the building of the application and why they were used. It will also cover how bugs and errors picked up by the testing were overcome. The evaluation chapter \ref{sec:eval} on page \pageref{sec:eval} will cover what the outcome of the product was like compared with the original idea. It will also cover feedback from users about their and how that feedback will be acted upon. The conclusion chapter \ref{sec:conc} on page \pageref{sec:conc} will wrap up the paper with what knowledge has been gained from this build, what would have been done differently, and also possible future outcomes from the product.\\

While there are already a number of options for monitoring your fitness, this application is different in a number of ways. It will be more focused than other application. The literature review will go into more detail on this subject but the main differences are it is focus points and the design. A lot of the applications are bloated with features, cluttering the application and distracting the user from their goals. The design of my application will help the user stay focused on their goals by having three main focus points and a minimalistic design approach.\\

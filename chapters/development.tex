This chapter will discuss the development stage of the production of the progress monitoring web application. It will cover a number of topics, from development tools used during this stage to development methods uses whilst building this application. It will also walk through, step by step, of the building process highlighting problems and how they were overcome.\\

\subsection{Tools}
A number of tools were used during development, some were used more and others but all of them played a key role in development. The main tool that was used is Sublime Text 2 \citep{sublimetext:2011}. This was the text editor that was used to build both the static mock up of the site and the full functioning web application. Sublime Text was chosen because of it's vast amount of plugins. This makes it very versatile and efficient when it comes to building productions applications. There were a number of other editors considered for the job, for example; Vim \citep{vim:2013} was going to be chosen on due to its rapid development capabilities from all of the navigation being done on the keyboard. But it was not chased due to the steep learning curve as this would eat into development time. The time investment into learning Vim would probably not be beneficial to such a small project but will be a consideration for future projects. During the development of the static site, a JavaScript task runner called Grunt. It was used to compile Sass, minify javaScript, and optimise images. When development moved onto Rails it was dropped as Rails compiles Sass and minifies JavaScript in the assists pipeline. The second main tool that was used was iTerm 2 \citep{iterm:2011} using the oh-my-zsh \citep{zsh:2009}. Using the command when building a Ruby on Rails application is absolutely necessary. It's used to generate controllers and models, start the Rails server, interact with the database, and a number of other key features. The terminal emulator is not really that important but the use of the Zsh shell was very useful to the development stage. It has a number of excellent features like displaying the current Git \citep{git:2014} branch that I was working in and if there was any changes that haven't been committed and also highlighting in the terminal. Seeing as a lot of the time was spent in the terminal, it was important to make that time as productive and as efficient as possible. Git was another tool that really helped with development. Git is a version control application which allows the user to commit versions of code so that it would be easy to revert back to a previous version of code if something went wrong with development. A slightly adapted branching methodology was used called Gitflow which allowed easier version control during development and having a live and stable master branch that could be used to push to Heroku. Heroku was used to host the site, more information on this service can be found in the deployment section \ref{sec:deployment} of this chapter.

\subsection{Deployment}
\label{sec:deployment}


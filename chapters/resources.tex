\section{Introduction}
This chapter will talk about all of the resources that were used during the development of Recur. It will go into detail about the languages used, which applications were used, and also what RubyGems were used. It will talk about why each was used against its competitors.

\section{Languages and Frameworks}
\subsection{Ruby on Rails}
Ruby on Rails was the main backbone to this application. It was chosen for its rapid development characteristics, its extendibility with the mass amount of Gems that can be easily added into a project, and its scalability when it comes under heavy load with lots of traffic. It was also chosen on personal preference as compared with other languages that have similar frameworks. There is not much difference between a framework written in Ruby and one that is written in PHP. They are both going to do the same job but someone will find it easier to write in Ruby and others will find it easier to write in PHP. The main point is the use of a framework at all. MVC frameworks provide a really good backbone to any application. The MVC framework is comprised of three components; the model, the view, and the controller. Each component has a specific task to take care of. In Ruby on Rails's case, the model is what talks to the database. It can enforce rules on the data coming into the database. For example; it must contain an email that is valid against a certain regular expression. The controller is the logic behind the application. It contains functions that link to the views of the application which will display the information. Framworks also allow for rapid development of applications by using a system called CRUD \nomenclature{CRUD}{Create, Read, Update, Destroy}. CRUD stands for create, read, update, and destroy. Each controller in Rails will use these four methods to create, read, update, and destroy a record of the model in the database. Instead of having to make your own CRUD system, Ruby on Rails gives you a number of calls that does it all for you. This is much easier and quicker than writing it all yourself.\\

\subsection{JavaScript and jQuery}
jQuery is a JavaScript library that allows developers to do very complex JavaScript functions with very little code. jQuery makes it really easy to interact and manipulate the web page. This means we can show and hide elements based on what the user has input in a box, or checked off in a check box. The main use for jQuery was the hiding of elements in forms and showing the menu on small screen devices. It was also used for the chart switching on the fitness log. This will be explored in more detail later in this chapter. jQuery is also used for its AJAX method. AJAX is really useful when you need to update a database without refreshing the page. This was used when checking off goals. When it comes to alternatives to JavaScript and jQuery there are none. JavaScript is quite unique in that it is the only language that allows developers to manipulate the webpage while it is in use. With the progression to CSS3, developers can now do some page manipulation with that using element selectors and change style attributes on click and hover, but not to the extent that jQuery can.\\

\section{Tools}
A number of tools were used during development, some were used more than others but all of them played a key role in development. The main tool that was used is Sublime Text 2 \citep{sublimetext:2011}. This was the text editor that was used to build both the static mock up of the site and the fully functioning web application. Sublime Text was chosen because of its vast amount of plugins. This makes it very versatile and efficient when it comes to building productions applications. There were a number of other editors considered for the job, for example; Vim \citep{vim:2013} was going to be chosen due to its rapid development capabilities from all of the navigation being done on the keyboard. But it was not chosen due to the steep learning curve (as this would eat into development time). The time invested into learning Vim would probably not be beneficial to such a small project, but could be considered for future projects.\\

During the development of the static site, a JavaScript task runner called Grunt was used to compile Sass \citep{sass:2006}, minify javaScript, and optimise images. When development moved onto Rails Grunt was dropped, as Rails compiles Sass and minifies JavaScript in the assets pipeline.\\

The second main tool that was used was iTerm 2 \citep{iterm:2011} using the oh-my-zsh \citep{zsh:2009}. Using the command when building a Ruby on Rails application is absolutely necessary. It is used to generate controllers and models, start the Rails server, interact with the database, and a number of other key features. The terminal emulator is not really that important but the use of the Zsh shell was very useful to the development stage. It has a number of excellent features like displaying the current Git \citep{git:2014} branch that I was working in and if there was any changes that have not been committed and also highlighting in the terminal. Seeing as a lot of the time was spent in the terminal, it was important to make that time as productive and as efficient as possible.\\

Git was another tool that really helped with development. Git is a version control system. It allows the developer to commit files, and revert back any changes if something goes wrong. A slightly adapted branching methodology was used called Gitflow \citep{gitflow:2010} which allowed easier version control during development and having a live and stable master branch that could be used to push to Heroku.\\

Heroku \citep{heroku:2007} was used to host the site, more information on this service can be found in the deployment section \ref{sec:deployment} of this chapter on page \pageref{sec:deployment}.\\

An application called Dash \citep{dash:2014} was also used during development. Dash allows you to download documentation for certain languages and frameworks and stores it offline. It allowed the Rails documentation to always be readily available whenever.

\section{Gems}
RubyGems \citep{rubygems:2009} is a package manager for the distribution of Gems. A Gem is a small package of Ruby code designed to either work on its own, for example a to-do list keeper, or work inside another application, for example the LazyHighCharts gem that was used to add charts to the log section of the website. They can be very useful by cutting down the amount of code needed to be written or by already having complex function built in and therefore not having to write them yourself. A number of other Gems were used during development, along side the default Rails Gems. The bcrypt-ruby gem was used for user creation and authentication. It allows you to create a password digest for storing in the user's record using one way encryption. This makes the system a lot more secure than storing the password in plain text, or even using a salt and hash to store the password. Sass-rails was used so that the style sheets could be written in Sass and compile to CSS at build time. Sass has a number of benefits over vanilla CSS, like being able to use variables for colours to allow for easier mass colour change. The Better Errors gem was also used during development to give a more in-depth error message when something went wrong. The default Rails error message is as descriptive as the better errors Gem so it becomes a lot easier to debug the application.\\

A number of testing gems were used during development to ensure the application was secure and working correctly. Rspec is the main gem which is used to write and run all of the tests. Using Rspec in conjunction with Capybara and FactoryGirl, it is easy to write extremely powerful tests. These tests not only interact with the Rails backend, but interact with pages and makes sure the redirects are working as intended.

\section{Conclusion}
To conclude this chapter, it is necessary to state that the outcome of the product will not be defined by the tools you use. Most tools are very similar and it almost always down to personal preference. In some circumstances, some tools will be better for a specific task than others, however, most tools are very similar and it is almost always down to personal preference.
\subsection{Introduction}
It is known that being overweight can be a risk to your health \citep{theordre:1985}. This, along with other factors, is driving a percentage of the overweight population to try and lose weight. Most people give up on weight loss due to lack of motivation and that is what my application is going to try and fix. By allowing users to track progress through an easy to use web application, that accessible anywhere via the internet, it will help keep the user motivated to keep losing the weight, and also keep the weight off once they have reached their goal weight.

There are already a number of weight tracking, fitness, and calorie counting applications out there on the market. Some of them offer all three of these services and a lot of other features, while other are more stripped down and only offer one or two. My application will focus on two or three areas too keep the user on track and not confuse them with things they do not need to worry about. The areas that my application will be focusing on are weight, goals and a fitness log.

\subsection{Related work}
This section will analyse a number of related websites offering a similar service to my one. It will give a brief explanation about the service and a break down of the good points and the bad.

\subsubsection{Fitocracy}
The first application is Fitocracy \citep{fitocracy:2007}, a web and mobile application to monitor fitness. It is heavily featured, which is a plus for some people, but others can find it distracting. One of its main features is its social network site. It has a social network backend that will let you follow your friends and give them kudos for completing certain exercises. This feature is excellent as it gives what is called a feedback loop. This is commonly used in mobile application today by having the ability to login with a social network and share progress to that social network. A social network backend is a great motivational tool but due the aim of keeping my application as simple as possible, I will not be implementing this. Fitocracy is very motivational based with things like challenges, quests, duels, and leaderboards all creating feedback loops you are getting constant rewards and recognition of your achievements. Again, my application will be staying very minimal so I won’t be implementing such massive features. Instead, my application shall just have goals, which will create a small feedback loop, giving the user motivation and recognition of their goals. The main downside to Fitocracy would be the overwhelming number of features. It has so many features and options all packed into the one screen it is very over whelming and not a nice experience to use. This is the basis of my application. To be stripped down bare of the nonsense that surrounds such applications like Fitocracy and focus on a few key points.


\subsubsection{Fitbit}
Fitbit is one of the biggest fitness apps out to date. What makes it so special is that you can purchase a wristband that monitors your activity and sleep, and then wireless syncs with your statistics with your Fitbit account. This is one of many items, known as wearable technology, that can track your activity. Others include the JawboneUp \citep{jawbone:2011} and the Nike+ Fuel Band \citep{nikefuelband:2012}. This allows for less manual data entry. It counts how many steps you have taken, how many calories burnt, and how many hours sleep. Its downside is that you can only specify certain exercises. For example, working out in the gym, you cannot tell it how much you are lifting, so it cannot give you an accurate reading on how many calories you have burnt.

Fitbit focuses on balancing your caloric intake and expenditure to help you hit your goal weight. After inputing your details and your goal weight, it will give you a calorie intake goal. It then balances your caloric intake, by tracking your diet, and your caloric expenditure, by tracking your activity. It will then match the two and show you your calories in versus out. Focusing on calories in versus calories out is another big way to lose weight. By decreasing your caloric intake and increase your caloric expenditure, you will lose weight.

Fitbit is very a well designed web application. It is taking the popular minimalist and flat design approach. It has a really clean dashboard that consists of six cards with information right there on them. If you click on one of the cards, you will see a modal pop up with more information on that card. In the bottom right hand corner of the pop up is an arrow, clicking the arrow will take you to even more information about that specific element of the application and this page will also allow you to enter data. This is a really efficient way of laying out an application as it give the user all the information on the first page, and more information and the ability to change that information is only two clicks away.

\subsubsection{Nike+ FuelBand}
The Nike+ FuelBand app is a little bit different from all of the others. It uses its own measurement (Nike+ fuel) to monitor your activity. You set yourself a goal and then as you move throughout the day, your fuel will go up. It connects to your mobile device via bluetooth so you can see your stats on the mobile application, and that syncs the data with their servers so you can see even more data on the web application. Their main goal was similar to mine, to keep the user motivated to get out and get exercising. They use the NikeFuel as a measurement as it is a universal way to measure all kinds of activity \citep{fuelbandpress:2013}. You can also add your friends and see how they are doing with their goals. This adds yet more motivation by being able to compete with your friends. The ability to connect with your friends is a great motivational tool but it takes away from the simplicity that my application is aiming for.

\subsection{Conclusion}
To conclude, my website will follow a similar simple, easy to use, design to Fitbit, but with a more focused aim on motivating the user to get back into the gym. Out of the two main types of wait loss, it will focusing on the fitness side. The decision was made to not implement both sides because to split the focus over two things would break the motivation from one focus to another. Also, the implementation of having calorie tracking would either be a big thing with a database of good types and their caloric value or having to have the user enter both the food type and then the caloric value of that good type. This can lead to an incorrect caloric reading.

After seeing some of the over complicated fitness applications with very poor user experience, the applications design was to be kept as simple as possible. Cutting down on user options and features while maintaining core, desirable, features is how it this will be achieved. Being a content driven application, the information needed to be readily available on the first page that the user sees, so that if they just wanted a quick view on how they are doing it would all be there as soon as they open the page.
